\documentclass[a4paper, twocolumn, superscriptaddress, longbibliography]{revtex4-2}
\usepackage[colorlinks,urlcolor=blue,linkcolor=blue]{hyperref}
\usepackage[utf8]{inputenc}
\usepackage{graphicx}
\usepackage{geometry}
\usepackage{amsmath}
\usepackage{amssymb}
\usepackage{amsfonts}
\usepackage{hyperref}
\usepackage{dsfont}
\usepackage[dvipsnames]{xcolor}
\usepackage{appendix}

\hypersetup{citecolor=cyan}
\geometry{hmargin=2cm,vmargin=2cm}

\newcommand{\vra}{\rangle \! \rangle}
\newcommand{\vla}{\langle \! \langle}
\newcommand{\hh}[1]{\hat{\hat{#1}}}


\begin{document}
	\author{Maxime Debertolis}
	%\email{maxime.debertolis@uni-bonn.de}
	\affiliation{Institute of Physics, University of Bonn, Nu\ss allee 12, 53115 Bonn, Germany}
	\title{Gaussian augmented Tensor Networks}

	\begin{abstract}

	\end{abstract}

	\maketitle

	\section{Introduction}
	Ansatz mixing a basis of orbitals to an MPS:
	\begin{equation}
		|\psi \rangle = \sum_{i=1}^{2^{L}} \nu_i^{} \hat{d}_i^{} |\psi_{\mathrm{G}}^{}\rangle,
	\end{equation}
	Gaussian state defined through fermionic operators:
	\begin{equation}
		|\psi_{\mathrm{G}}\rangle = \prod_{j=1}^{L}\left(\hat{c}_{j}^{\dagger}\right)^{n_j}|0\rangle,
	\end{equation}
	set of destabilizers:
	\begin{equation}
		\hat{d}_i = \prod_{j=1}^{L}\hat{\gamma}_{2j}^{b_{i}(j)},
	\end{equation}
	in which $b_i$ is the binary representation of $i$ such that $b_{i}(j) \in \{0,1\}$ tells if the occupation of the orbital $j$ in the state to which the product of \emph{destabilizers} is applied will change. The set of state generated by the combination of the Gaussian state and the destabilizers form an orthonormal basis of Gaussian states for the Hilbert space of dimension $2^{L}$, since there are $2^L$ combination of the destabilizers, $\hat{d}_j |\psi_{\mathrm{G}}\rangle$ is also a Gaussian state and $\langle \psi_{G}|\hat{d}_i \hat{d}_j |\psi_{\mathrm{G}}\rangle = \delta_{ij}$.
	Majorana operators and fermionic operators are related as:
	\begin{equation}
		\hat{\gamma}_{2j-1} = \hat{c}^{\dagger}_{j} + \hat{c}_{j}, \;\;
		\hat{\gamma}_{2j} = -i ( \hat{c}^{\dagger}_{j} + \hat{c}_{j})
	\end{equation}
	Basis of majoranas evolve after Gaussian operations:
	\begin{equation}
		\hat{\tilde{\gamma}}_j = \sum_{k=1}^{2L} R_{jk}\hat{\gamma}_{k}
	\end{equation}
	The basis of natural orbitals are reconstructed through evolved majoranas:
	\begin{equation}
		\hat{\tilde{c}}_{j} = \frac{1}{2}\left(\hat{\tilde{\gamma}}_{2j-1} + i\,\hat{\tilde{\gamma}}_{2j}\right),
	\end{equation}
	which form a set of \emph{stabilizers} that allows with the corresponding natural occupations to fully determine a Gaussian state.
	\begin{itemize}
		\item {\bf Gaussian operations}: Update the basis of orbitals $\hat{\tilde{\gamma}}_j = \sum_{k=1}^{2L} R_{jk}\hat{\gamma}_{k}$. We simply follow the current basis of orbitals by multiplying the successive rotations after each Gaussian operation to each other, such that the frame and thus the basis of orbital is always known.
		\item {\bf Non-Gaussian operations}: The combination of odd majoranas and even ones (our \emph{destabilizers}) form together a complete basis for operators, such that $\hat{U} = \sum_{i} \phi_{i} \hat{s}_{u_i}\hat{d}_{v_i}$, which for a local gate acting on nearest neighbors acts as a rotation, which can be efficiently implemented through a quantum circuits. For non-local operations that are non-local in the current basis of orbitals, we can represent an MPO with at most bond dimension $L^{4}$ for exponential of operators that are quartic in the majorana modes. 
		\item {\bf Expectation values}: Expectation values can simply be performed by decomposition of the observable in the current basis that is to be computed: $\langle \psi | \hat{O} | \psi \rangle = \sum_{k=1}^{L^{r}} a_{k}\langle \nu | \hat{O}_k | \nu \rangle$, where $a_k$ is the corresponding weight of the operator $\hat{O}_{k}$ and $r$ is the number of fermionic modes encoded in the observables. For a density operator $n_\alpha$, the decomposition follows $n_\alpha = \sum_{ij} R^{*}_{i\alpha} R_{\alpha j} \hat{\tilde{c}}^{\dagger}_{i}\hat{\tilde{c}}^{}_{j}$, such that $a_{ij} = R^{*}_{i\alpha} R_{\alpha j}$.
	\end{itemize}

	\section{Conclusion and outlook}

	\textit{Acknowledgement}. The author thank Neil Dowling and Julien Bréhier for discussions. The author was supported by the Deutsche Forschungsgemeinschaft through the cluster of excellence ML4Q (EXC 2004, project-id 390534769) and by the Deutsche Forschungs- gemeinschaft through CRC 1639 NuMeriQS (project-id 511713970).

	\bibliography{Biblio.bib}

           \begin{appendices}

	\end{appendices}


\end{document}

